% !TeX spellcheck = de_DE
\documentclass[11pt,a4paper]{article}

\usepackage[left=2cm, text={17cm,24cm}, top=3cm]{geometry}
\usepackage[utf8]{inputenc}
\usepackage[czech]{babel}
\usepackage{times}
\begin{document}
	\begin{titlepage}
		\begin{center}
			\Huge
			\textsc{Vysoké učení technické v Brně\\[-2mm]
			\huge Fakulta informačních technologií}
		
			\vspace{\stretch{0.382}}
			
			\LARGE
			Typografie a publikování\,-- 3. projekt\\[-0.5mm]
			\Huge Tabulky a obrázky
			\vspace{\stretch{0.618}}
		\end{center}
		{\Large 29. března 2017 \hfill Martin Omacht}
	\end{titlepage}
	
	\section{Úvodní strana}
	Název práce umístěte do zlatého řezu a nezapomeňte uvést dnešní datum a vaše jméno a příjmení.
	
	\section{Tabulky}
	Pro sázení tabulek můžeme použít buď prostředí \texttt{tabbing} nebo prostředí \texttt{tabular}.

	\subsection{Prostředí \texttt{tabbing}}
	Při použití \texttt{tabbing} vypadá tabulka následovně:
	
	Toto prostředí se dá také použít pro sázení algoritmů, ovšem vhodnější je použít prostředí \texttt{algorithm} nebo \texttt{algorithm2e} (viz sekce \ref{algoritmy}).

	\subsection{Prostředí \texttt{tabular}}
	Další možností, jak vytvořit tabulku, je použít prostředí tabular. Tabulky pak 
	budou vypadat takto\footnote{Kdyby byl problem s cline, zkuste se podívat třeba sem: http://www.abclinuxu.cz/tex/poradna/show/325037}:

	\section{Algoritmy} \label{algoritmy}
	Pokud budeme chtít vysázet algoritmus, můžeme použít prostředí \texttt{algorithm}\footnote{Pro nápovědu, jak zacházet s prostředím algorithm, můžeme zkusit tuhle stránku: http://ftp.cstug.cz/pub/tex/CTAN/macros/latex/contrib/algorithms/algorithms.pdf.} nebo \texttt{algorithm2e}\footnote{Pro algorithm2e zase tuhle: http://ftp.cstug.cz/pub/tex/CTAN/macros/latex/contrib/algorithm2e/algorithm2e.pdf.}. Příklad použití prostředí \texttt{algorithm2e} viz Algoritmus ....

	\section{Obrázky}
	Do našich článků můžeme samozřejmě vkládat obrázky. Pokud je obrázkem fotografie,
	můžeme klidně použít bitmapový soubor. Pokud by to ale mělo být nějaké schéma nebo něco podobného, je dobrým zvykem takovýto obrázek vytvořit vektorově.
\end{document}