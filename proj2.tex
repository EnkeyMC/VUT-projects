\documentclass[11pt, a4paper, twocolumn]{article}

\usepackage[left=1.5cm, text={18cm,25cm}, top=2.5cm]{geometry}
\usepackage[utf8]{inputenc}
\usepackage[czech]{babel}
\usepackage[IL2]{fontenc}
\usepackage{times}
\usepackage{amsthm}
\usepackage{amssymb}
\usepackage{dsfont}

\theoremstyle{definition}
\newtheorem{definice}{Definice}[section]

\theoremstyle{plain}
\newtheorem{algoritmus}{Algoritmus}[section]
\newtheorem{veta}{Věta}

\begin{document}
	\begin{titlepage}
		\begin{center}
			{\linespread{0.75}
				\Huge
				\textsc{Fakulta informačních technologií\\ Vysoké učení technické v Brně}
				\vspace{\stretch{0.382}}
			}

			{\linespread{0.8}
				\LARGE
				Typografie a publikování - 2. projekt\\
				Sazba dokumentů a matematických výrazů
				\vspace{\stretch{0.618}}
			}
		\end{center}		
		{\Large 2017 \hfill Martin Omacht}
	\end{titlepage}

	\section*{Úvod} % (fold)
	\label{sec:úvod}
	V této úloze si vyzkoušíme sazbu titulní strany, matematických vzorců, prostředí a dalších textových struktur obvyklých pro technicky zaměřené texty (například rovnice ... nebo definice ... na straně ...).

	Na titulní straně je využito sázení nadpisu podle optického středu s využitím zlatého řezu. Tento postup byl probírán na přednášce.
	% section úvod (end)

	\section{Matematický text} % (fold)
	\label{sec:matematický_text}
	Nejprve se podíváme na sázení matematických symbolů a výrazů v plynulém textu. Pro množinu $V$označuje $\mathrm{card}(V)$ kardinalitu $V$.
	Pro množinu $V$ reprezentuje $V^\ast$ volný monoid generovaný množinou $V$ s operací konkatenace.
	Prvek identity ve volném monoidu $V^*$ značíme symbolem $\varepsilon$.
	Nechť $V^+ = V^\ast - \{\varepsilon\}$. Algebraicky je tedy $V^\ast$ volná pologrupa generovaná množinou $V$ s operací konkatenace.
	Konečnou neprázdnou množinu $V$ nazvěme \emph{abeceda}.
	Pro $\omega \in V^\ast$ označuje $|\omega|$ délku řetězce $\omega$. Pro $W \subseteq V$ označuje $\mathrm{occur}(w,W)$ počet výskytů symbolů z $W$ v řetězci $w$ a $\mathrm{sym}(w,i)$ určuje $i$-tý symbol řetězce $w$; například $\mathrm{sym}(abcd,3)=c$.

	Nyní zkusíme sazbu definic a vět s využitím balíku \texttt{amsthm}.

	\begin{definice}
	\emph{Bezkontextová gramatika} je čtveřice $G=(V,T,P,S)$, kde $V$ je totální abeceda,
	$T \subseteq V$ je abeceda terminálů, $S \in (V-T)$ je startující symbol a $P$ je konečná množina \emph{pravidel} tvaru $q: A \rightarrow \alpha$, kde $A \in (V - T)$, $\alpha \in V^*$ a $q$ je návěští tohoto pravidla. Nechť $N = V - T$ značí abecedu neterminálů.
	Pokud $q: A \rightarrow \alpha \in P$, $\gamma,\delta \in V^*$, $G$ provádí derivační krok z $\gamma A \delta$ do $\gamma \alpha \delta$ podle pravidla $q: A \rightarrow \alpha$, symbolicky píšeme $\gamma A \delta \Rightarrow \gamma \alpha \delta [q: A \rightarrow \alpha]$ nebo zjednodušeně $\gamma A \delta \Rightarrow \gamma \alpha \delta$. Standardním způsobem definujeme $\Rightarrow^m$, kde $m \geq 0$. Dále definujeme tranzitivní uzávěr $\Rightarrow^+$ a tranzitivně-reflexivní uzávěr $\Rightarrow^*$.
	\end{definice}

	Algoritmus můžeme uvádět podobně jako definice textově, nebo využít pseudokódu vysázeného ve vhodném prostředí (například \texttt{algorithm2e}).

	\begin{algoritmus}
	Algoritmus pro ověření bezkontextovosti gramatiky. Mějme gramatiku G = (N, T, P, S).
		\begin{enumerate}
	 		\item \label{prvniKrok} Pro každé pravidlo $p \in P$ proveď test, zda $p$ na levé straně obsahuje právě jeden symbol z $N$.
	 		\item Pokud všechna pravidla splňují podmínku z kroku \ref{prvniKrok}, tak je gramatika $G$ bezkontextová.
	 	\end{enumerate}
	\end{algoritmus}

	\begin{definice}
	Jazyk definovaný gramatikou $G$ definujeme jako $L(G) = {w \in T^*|S \Rightarrow^* w}$.
	\end{definice}

	\subsection{Podsekce obsahující větu}

	\begin{definice}
	Nechť $L$ je libovolný jazyk. $L$ je \emph{bezkontextový jazyk}, když a jen když $L = L(G)$, kde $G$ je libovolná bezkontextová.
	\end{definice}

	\begin{definice}
	Množinu $\mathcal{L}_{CF} = \{L|L$ je bezkontextový jazyk$\}$ nazýváme \emph{třídou bezkontextových jazyků}.
	\end{definice}

	\begin{veta} \label{veta1}
	Nechť $L_{abc} = \{a^n b^n c^n | n \geq 0\}$. Platí, že $L_{abc} \notin \mathcal{L}_{CF}$.
	\end{veta}

	\begin{proof}
	Důkaz se provede pomocí Pumping lemma pro bezkontextové jazyky, kdy ukážeme, že není možné, aby platilo, což bude implikovat pravdivost věty \ref{veta1}.
	\end{proof}
	% section matematický_text (end)

	\section{Rovnice a odkazy}
	Složitější matematické formulace sázíme mimo plynulý text. Lze umístit několik výrazů na jeden řádek, ale pak je třeba tyto vhodně oddělit, například příkazem \verb|\quad|. 

	$$\sqrt[x^2]{y^3_0} \quad \mathbb{N} = \{0,1,2,\ldots\} \quad x^{y^y} \neq x^{yy} \quad z_{i_j} \not\equiv z_{ij}$$

	V rovnici (...) jsou využity tři typy závorek s různou explicitně definovanou velikostí.

	\begin{eqnarray}
		\left\{ {\left[ (a+b)*c \right]}^d + 1 \right\} = x
	\end{eqnarray}

	V této větě vidíme, jak vypadá implicitní vysázení limity ... v normálním odstavci textu. Podobně je to i s dalšími symboly jako ... či ... . V případě vzorce ... jsme si vynutili méně úspornou sazbu příkazem \verb|\limits|.

	...
	% section rovnice_a_odkazy (end)
\end{document}