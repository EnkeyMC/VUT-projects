\documentclass[11pt, a4paper, twocolumn]{article}

\usepackage[left=1.5cm, text={18cm,25cm}, top=2.5cm]{geometry}
\usepackage[utf8]{inputenc}
\usepackage[czech]{babel}
\usepackage[IL2]{fontenc}
\usepackage{times}
\usepackage{amsthm}

\theoremstyle{definition}
\newtheorem{definice}{Definice}[section]

\begin{document}
	\begin{titlepage}
		\begin{center}
			{\linespread{0.75}
				\Huge
				\textsc{Fakulta informačních technologií\\ Vysoké učení technické v Brně}
				\vspace{\stretch{0.382}}
			}

			{\linespread{0.8}
				\LARGE
				Typografie a publikování - 2. projekt\\
				Sazba dokumentů a matematických výrazů
				\vspace{\stretch{0.618}}
			}
		\end{center}		
		{\Large 2017 \hfill Martin Omacht}
	\end{titlepage}

	\section*{Úvod} % (fold)
	\label{sec:úvod}
	V této úloze si vyzkoušíme sazbu titulní strany, matematických vzorců, prostředí a dalších textových struktur obvyklých pro technicky zaměřené texty (například rovnice ... nebo definice ... na straně ...).

	Na titulní straně je využito sázení nadpisu podle optického středu s využitím zlatého řezu. Tento postup byl probírán na přednášce.
	% section úvod (end)

	\section{Matematický text} % (fold)
	\label{sec:matematický_text}
	Nejprve se podíváme na sázení matematických symbolů a výrazů v plynulém textu. Pro množinu $V$označuje $\mathrm{card}(V)$ kardinalitu $V$.
	Pro množinu $V$ reprezentuje $V^\ast$ volný monoid generovaný množinou $V$ s operací konkatenace.
	Prvek identity ve volném monoidu $V^*$ značíme symbolem $\varepsilon$.
	Nechť $V^+ = V^\ast - \{\varepsilon\}$. Algebraicky je tedy $V^\ast$ volná pologrupa generovaná množinou $V$ s operací konkatenace.
	Konečnou neprázdnou množinu $V$ nazvěme \emph{abeceda}.
	Pro $\omega \in V^\ast$ označuje $|\omega|$ délku řetězce $\omega$. Pro $W \subseteq V$ označuje $\mathrm{occur}(w,W)$ počet výskytů symbolů z $W$ v řetězci $w$ a $\mathrm{sym}(w,i)$ určuje $i$-tý symbol řetězce $w$; například $\mathrm{sym}(abcd,3)=c$.

	Nyní zkusíme sazbu definic a vět s využitím balíku \texttt{amsthm}.

	\begin{definice}
	\emph{Bezkontextová gramatika} je čtveřice $G=(V,T,P,S)$, kde $V$ je totální abeceda,
	$T \subseteq V$ je abeceda terminálů, $S \in (V-T)$ je startující symbol a $P$ je konečná množina \emph{pravidel} tvaru $q: A \rightarrow \alpha$, kde $A \in (V - T)$, $\alpha \in V^*$ a $q$ je návěští tohoto pravidla. Nechť $N = V - T$ značí abecedu neterminálů.
	Pokud $q: A \rightarrow \alpha \in P$, $\gamma,\delta \in V^*$, $G$ provádí derivační krok z $\gamma A \delta$ do $\gamma \alpha \delta$ podle pravidla $q: A \rightarrow \alpha$, symbolicky píšeme $\gamma A \delta \Rightarrow \gamma \alpha \delta [q: A \rightarrow \alpha]$ nebo zjednodušeně $\gamma A \delta \Rightarrow \gamma \alpha \delta$. Standardním způsobem definujeme $\Rightarrow^m$, kde $m \geq 0$. Dále definujeme tranzitivní uzávěr $\Rightarrow^+$ a tranzitivně-reflexivní uzávěr $\Rightarrow^*$.
	\end{definice}

	Algoritmus můžeme uvádět podobně jako definice textově, nebo využít pseudokódu vysázeného ve vhodném prostředí (například algorithm2e).

	Algoritmus: Algoritmus pro ověření bezkontextovosti gramatiky. Mějme gramatiku G = (N, T, P, S).
	 * Pro každé pravidlo ... proveď test, zda ... na levé straně obsahuje právě jeden symbol z ... .
	 * Pokud všechna pravidla splňují podmínku z kroku ..., tak je gramatika ... bezkontextová.

	Definice: Jazyk definovaný gramatikou ... definujeme jako ... .

	1.1 Podsekce obsahující větu

	Definice: Nechť ... je libovolný jazyk. ... je bezkontextový jazyk, když a jen když ..., kde ... je libovolná bezkontextová gramatika.

	Definice: Množinu ... nazýváme třídou bezkontextových jazyků.

	Věta: Nechť .... Platí, že ....

	Důkaz: Důkaz se provede pomocí Pumping lemma pro bezkontextové jazyky, kdy ukážeme, že není možné, aby platilo, což bude implikovat pravdivost věty ... .
	% section matematický_text (end)
\end{document}